\chapter{国内外研究现状}
\par 6自由度物体位姿估计问题是指通过传感器数据计算物体坐标系相对于相机坐标系的位姿变换。位姿变换包括平移分量和旋转分量,能用一个 $4\times 4$ 的位姿变换矩阵描述。传感器数据通常包括单目RGB相机、深度相机或者多目RGB相机或者多个视角的相机组合。物体坐标系和相机坐标系分别定义在物体的模型和相机上。物体的模型通常是一个三维模型,可以是点云、网格(mesh)模型或者CAD模型,通常具有纹理或颜色信息。相机的模型通常是一个消除了畸变的针孔相机模型,相机内参 $K$ 已知。
\par 刚性物体在空间中的位置和姿态具有六个自由度(6 Degree of Freedom,又称6D),其中三个平移自由度,用笛卡尔坐标系中的坐标表示;三个旋转自由度,可以用欧拉角、四元数、旋转矩阵等方式进行表示。物体的6D位姿估计就是通过传感器获取场景图像、点云等信息,最终获得场景中目标物体在相机坐标系或者世界坐标系下的位置和姿态。
\par 传感器可以采用RGB彩色相机,RGB-D彩色深度相机或者双目立体相机,不同的传感器配置具有不同的特点。RGB彩色相机成本低,易于部署,但用于物体位姿估计的难度最大;RGB-D彩色深度相机的成本较高,且深度相机在帧率、视场角、深度测量范围的限制,使得深度相机不适用于高反光物体、透明物体或者快速运动的物体,但深度信息能够极大地提高位姿估计的精度,用于物体位姿估计的难度最低,目前大部分移动设备都未配备深度相机;双目立体相机能够用双目获取深度信息,但双目之间的距离限制了深度信息的精度,且目前数据集较少,相关研究也比较少。如何仅使用RGB彩色相机实现高精度的位姿估计,是目前的研究热点。

以下相关研究仅围绕实例级物体位姿估计展开,不涉及类别级物体位姿估计的问题。
\section{基于RGB的物体位姿估计}

\section{基于RGB-D的物体位姿估计}

\section{面向泛化的物体位姿估计}
