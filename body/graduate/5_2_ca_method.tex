\section{方法}

本节详细描述了我们提出的基于轮廓对齐的6D物体位姿估计的优化方法。该过程包括使用可见和完整掩码提取感兴趣轮廓(Contour of Interest,CoI),并基于CoI优化位姿估计。

\subsection{方法架构}

给定一个具有已知相机内参$K$的RGB图像和一组物体的CAD模型,我们的目标是估计每个物体的位姿。位姿通过旋转矩阵$\mathbf{R} \in SO(3)$和平移向量$\mathbf{t} \in \mathbb{R}^3$进行描述。

与其他需要两个独立网络的方法\cite{labbe2020cosypose}相比,我们的方法通过一个网络实现初始位姿估计和优化。整个过程的概述请参见\autoref{fig:ca_overview}。在初始过程中,我们通过检测器提取RGB图像中的RoI区域,接着通过编码网络得到物体的完整掩码,可见部分掩码以及其他特征。通过完整掩码和可见部分掩码,我们通过图形学处理的方式得到CoI。通过解码器可以得到物体的初始位姿$\mathbf{T}_0$。优化过程涉及迭代调整初始位姿$\mathbf{T}_0=[\mathbf{R}_0|\mathbf{t}_0]$,从而得到后续的位姿$\mathbf{T}_1$,$\mathbf{T}_2$,等等。

\begin{figure}[htbp]
    \centering
    \begin{overpic}[width=1.0\textwidth]{figure/ca/process.jpg}
        \put (10, 5) {RGB图像}
        \put (30, 3.5) {\scriptsize 其他特征}
        \put (30, 13) {\scriptsize 可见掩码}
        \put (30, 23) {\scriptsize 完整掩码}
        \put (2, 39) {检测器}
        \put (15, 28.5) {RoI}
        \put (29.5, 39) {\shortstack{编码器}}
        \put (25, 45) {初始过程}
        \put (47, 5) {\shortstack{解码器}}
        \put (50, 15) {$\mathbf{T}_0$}
        \put (50, 30) {$\mathbf{C}_\text{ren}$}
        \put (50, 40.5) {CoI}
        \put (73, 45) {优化过程}
        \put (71, 4.5) {$\mathbf{T}_n$}
        \put (66.5, 19) {迭代$n$}
        \put (66.5, 30.5) {迭代$2$}
        \put (66.5, 40) {迭代$1$}
        \put (81.5, 26) {$\mathbf{T}_2$}
        \put (81.5, 37) {$\mathbf{T}_1$}
        \put (92, 25) {$\mathbf{C}_\text{ren}$}
        \put (92, 36.5) {$\mathbf{C}_\text{ren}$}
    \end{overpic}
    \caption{方法概览}
    \label{fig:ca_overview}
\end{figure}

完整掩码$\mathbf{M}_\text{all}$表示整个物体的区域,不受遮挡影响,而可见掩码$\mathbf{M}_\text{vis}$仅表示图像中可见的物体部分。

在本研究中,我们使用像素坐标$\mathbf{p} = [u, v]$来描述图像中的像素位置。此外,3D物体的顶点表示为$\mathbf{P}=[X, Y, Z] \in \mathbb{R}^3$。为了简化表示,我们也使用相同的符号$\mathbf{P}$来表示其齐次形式,即$\mathbf{P}=[X, Y, Z, 1]$。默认情况下,一维向量被假定为列向量。

% 。 \ding{172}\enspace CoI; \ding{173}\enspace 由遮挡生成的轮廓; \ding{174}\enspace 由于遮挡不可见的轮廓。

\subsection{感兴趣轮廓}
在最近的研究中,可见掩码通常被用作辅助变量,通常表示为大小为$H \times W$的特征图,范围在0到1之间。通过拓展掩码输出层的参数,将输出层扩展为同时包含可见掩码和完整掩码,大小为$H \times W$。通过二值化可见掩码$\mathbf{M}_\text{vis}$和完整掩码$\mathbf{M}_\text{all}$,并使用OpenCV中的边缘提取算法\cite{itseez2015opencv},我们可以提取可见掩码的轮廓$\mathbf{C}_\text{vis}$以及完整掩码的轮廓$\mathbf{C}_\text{all}$。CoI定义为可见掩码轮廓$\mathbf{C}_\text{vis}$和完整掩码轮廓$\mathbf{C}_\text{all}$的交集。数学上,CoI可以表示为$\text{CoI} = \mathbf{C}_\text{vis} \cap \mathbf{C}_\text{all}$。为了确定可见轮廓$\mathbf{C}_\text{vis}$中的像素$\mathbf{p}_\text{vis}$是否属于CoI,我们迭代每个像素并检查它是否存在于$\mathbf{C}_\text{all}$中,过程如\autoref{fig:ca_CoI}所示,图中\ding{172}\enspace 为 CoI; \ding{173}\enspace 是由遮挡生成的轮廓; \ding{174}\enspace 是由于遮挡不可见的轮廓。对于可见轮廓$\mathbf{C}_\text{vis}$以及完整掩码的轮廓$\mathbf{C}_\text{all}$包含的像素坐标较少时,我们直接使用双重循环进行遍历,检测是否能够找到$(\mathbf{p}_\text{vis} - \mathbf{p}_\text{all})^2 < \delta^2$。其中$\delta$为阈值,此时算法复杂度为$O(n^2)$。若可见轮廓$\mathbf{C}_\text{vis}$以及完整掩码的轮廓$\mathbf{C}_\text{all}$包含像素过多,则可以先把完整轮廓$\mathbf{C}_\text{all}$通过二维的形式存储在图像中,通过遍历可见轮廓$\mathbf{C}_\text{vis}$坐标周围的像素点坐标是否有完整轮廓来实现$O(n)$的算法。

\begin{figure}[htbp]
    \centering
    \begin{overpic}[width=0.6\textwidth]{figure/ca/CoI.jpg}
        \put (2, 1) {\small 感兴趣轮廓}
        \put (3.5, 27) {\small 可见掩码}
        \put (3.5, 53) {\small 完整掩码}
    \end{overpic}
    \caption{感兴趣轮廓(CoI)生成过程}
    \label{fig:ca_CoI}
\end{figure}

\subsection{基于轮廓对齐的细化}

当提供物体模型、相机内参矩阵$K$和初始姿态$\mathbf{T}$用于细化过程的一次迭代时,我们可以根据姿态$\mathbf{T}$渲染物体模型的前景掩码。随后,我们从渲染图像中提取前景掩码的轮廓,表示为渲染轮廓$\mathbf{C}_\text{ren}$。

在我们的方法中,我们使用针孔相机模型将3D模型点投影到未失真的图像上。投影过程由以下方程定义:
\begin{equation}
\begin{bmatrix}u\\v\end{bmatrix} = \pi(\mathbf{P}_C) = \begin{bmatrix}\frac{X_C}{Z_C} f_x+c_x\\[3pt] \frac{Y_C}{Z_C} f_y+c_y\end{bmatrix}
% [u, v] = \pi(\mathbf{X})=[\frac{X}{Z} f_x+c_x, \frac{Y}{Z} f_y+c_y]
\label{eq:camera}
\end{equation}
在这个方程中,变量$f_x$和$f_y$对应焦距,而$c_x$和$c_y$表示主点的坐标。术语$\mathbf{P}_C$指的是相机坐标系中的顶点。

\par 模型坐标系中的顶点可以使用齐次形式转换到相机坐标系中,如下所示:
\begin{equation}
\mathbf{P}_C = \mathbf{T}\mathbf{P}_M=
\begin{bmatrix}
    \mathbf{R}&\mathbf{t}\\
    \mathbf{0} & 1
\end{bmatrix}
\mathbf{P}_M
\end{equation}
这里,$\mathbf{P}_M$表示在模型坐标系中表示的3D模型顶点。

我们使用向量$\mathbf{\theta}_r \in \mathbb{R}^3$表示旋转变化,使用向量$\mathbf{\theta}_t \in \mathbb{R}^3$表示平移变化。对于姿态的小变化,我们推导出更新方程如下:
\begin{equation}
\label{eq:update_rt}
\begin{bmatrix}
\mathbf{\hat{R}}&\mathbf{\hat{t}}\\
    \mathbf{0} & 1
\end{bmatrix}
=
\begin{bmatrix}
    \mathbf{R}&\mathbf{t}\\
    \mathbf{0} & 1
\end{bmatrix}
\begin{bmatrix}
    \mathbf{I} + [\mathbf{\theta}_r]_{\times}&\mathbf{\theta}_{t}\\
    \mathbf{0} & 1
\end{bmatrix}
\
\end{equation}
这里,$\hat{\mathbf{R}}$和$\hat{\mathbf{t}}$分别表示更新后的旋转和平移。术语$[\mathbf{\theta}_r]_{\times}$表示$\mathbf{\theta}_r$的反对称矩阵形式。
在三维空间中,任何向量$\mathbf{v} = [v_1, v_2, v_3]$都可以与一个反对称矩阵相关联,定义如下:
\begin{equation}
     [\mathbf{v}]_{\times} = \begin{bmatrix} 0 & -v_3 & v_2 \\ v_3 & 0 & -v_1 \\ -v_2 & v_1 & 0 \end{bmatrix} 
\end{equation}

使用更新方程\autoref{eq:update_rt},可以将3D模型点投影到相机坐标系中,如下所示:
\begin{equation}
    \mathbf{P}_C' = 
\begin{bmatrix}
\mathbf{\hat{R}}&\mathbf{\hat{t}}\\
    \mathbf{0} & 1
\end{bmatrix}
\mathbf{P}_M
\label{eq:update_pc}
\end{equation}
这里,$P_C'$表示相机坐标系中更新后的模型顶点。

对于每个点$\mathbf{p}_i=[p_{ui}, p_{vi}] \in \mathbf{C}_{pri}$,我们将在渲染轮廓$\mathbf{C}_{ren}$中找到最近的点$\mathbf{r}_i=[r_{ui}, r_{vi}]$。误差模型定义如下:
\begin{equation}
    \mathbf{e} = \sum_{i} e_i = \sum_{i} \left( \begin{bmatrix}r_{ui}\\r_{vi}\end{bmatrix}-\begin{bmatrix}p_{ui}\\p_{vi}\end{bmatrix} \right)
    \label{eq:error}
\end{equation}

通过求和每个$e_i$的导数,可以得到误差的导数。应用链式法则,可以表示为:

\begin{equation}
    \frac{\partial e_i}{\partial \delta \xi}=
\frac{\partial e_i}{\partial \mathbf{P}_C'} \frac{\partial \mathbf{P}_C'}{\partial \delta \xi}
\label{eq:dedxi}
\end{equation}
这里,$\xi=[\mathbf{\theta}_r \  \mathbf{\theta}_t] \in \mathbb{R}^6$。

第一项可以使用相机模型\autoref{eq:camera}和误差模型\autoref{eq:error}推导出来。为了简化公式,我们将省略下标$i$。

\begin{equation}
\frac{\partial e}{\partial \mathbf{P}_C'}=
\begin{bmatrix}
\frac{\partial r_{u}}{\partial X_C'}&\frac{\partial r_{u}}{\partial Y_C'}&\frac{\partial r_{u}}{\partial Z_C'}\\[5pt]\frac{\partial r_{v}}{\partial X_C'}&\frac{\partial r_{v}}{\partial Y_C'}&\frac{\partial r_{v}}{\partial Z_C'}\\ 
\end{bmatrix}
=
\begin{bmatrix}\frac{f_x}{Z_C'}&0&-\frac{f_x X_C'}{Z_C'^{2}}\\0&\frac{f_y}{Z'}&-\frac{f_y Y'}{Z_C'^{2}}\\ 
\end{bmatrix}
\label{eq:dedPC}
\end{equation}

第二项可以从更新方程\autoref{eq:update_rt}和\autoref{eq:update_pc}推导出来。

\begin{equation}
    \frac{\partial \mathbf{P}_C'}{\partial \mathbf{\xi}}=
\begin{bmatrix}
\mathbf{I} & -(\mathbf{P}_C')_{\times}
\end{bmatrix}
\label{eq:dPCdxi}
\end{equation}

\subsection{优化}

对于单步优化,使用带有Tikhonov正则化的牛顿优化计算更新向量,可以表示为:
\begin{equation}
\mathbf{\delta \xi} = 
  \left( 
    -\mathbf{H} + 
    \lambda \mathbf{I}_{6}
  \right)^{-1}\mathbf{g}
\label{eq:delta_si}
\end{equation}
在上面的方程中,
$\mathbf{g}$和$\mathbf{H}$分别表示梯度向量和Hessian矩阵,定义如下:
\begin{equation}
    \mathbf{g} = \sum_{i}\frac{\partial e_i}{\partial \delta \mathbf{\xi}}, \ \ \mathbf{H} = \mathbf{g} \mathbf{g}^T
\label{eq:gH}
\end{equation}
其中,$\lambda$是正则化参数。

我们算法的伪代码简要概述在\autoref{alg:refinement}中。

\begin{algorithm}[!ht]
    \caption{渲染和优化过程}
    \label{alg:refinement}
    \renewcommand{\algorithmicrequire}{\textbf{输入:}}
    \renewcommand{\algorithmicensure}{\textbf{输出:}}
    
    \begin{algorithmic}[1]
        % \REQUIRE $A$, $B$, $C$(This is Inputs)  %%input
        \REQUIRE $T_0$, $P_M$, $K$, $CoI$
        \ENSURE $T_n$, $e$    %%output
        \STATE 基于$T_0$渲染$M_\text{ren}$
        \STATE 提取$C_\text{ren}$
        \FOR {每个$it \in [1,10]$}
    \STATE {将CoI中的每个点与$C_\text{ren}$中的点匹配}
    \STATE {通过\autoref{eq:error}计算$e$}
    \STATE 检查是否提前停止
    \STATE {通过\autoref{eq:dedPC}计算$\frac{\partial e}{\partial \mathbf{P}_C'}$}
    \STATE {通过\autoref{eq:dPCdxi}计算$\frac{\partial \mathbf{P}_C'}{\partial \mathbf{\xi}}$}
    \STATE {通过\autoref{eq:dedxi}计算$\frac{\partial \mathbf{e}}{\partial \delta \xi}$}
    \STATE {通过\autoref{eq:delta_si}计算$\mathbf{\delta \xi}$}
    \STATE {通过\autoref{eq:update_rt}更新姿态$T_\text{it}$}
    \ENDFOR
        \RETURN 输出
    \end{algorithmic}
\end{algorithm}

\subsection{实现细节}

如\autoref{fig:ca_overview}所示,我们的方法基于上一章的SymNet网络生成初始姿态$\mathbf{T}_0$,并提供可见掩码$\mathbf{M}_\text{vis}$和完整掩码$\mathbf{M}_\text{all}$。首先,检测器分析RGB图像并提取感兴趣区域(RoI)。随后,编码器模块生成完整掩码、可见掩码以及解码器模块所需的其他特征。解码器模块生成初始位姿$\mathbf{T}_0$。在优化过程中,根据当前姿态迭代计算$\mathbf{C}_\text{ren}$并更新位姿参数。

优化过程中,我们使用CoI的所有点,总计约200个点。默认情况下,Tikhonov参数设置为$\lambda = 10^4$。采用提前停止策略,即当\autoref{eq:error}计算的误差不再减少时,终止优化。

值得注意的是,我们的方法可扩展至其他先进技术,如ZebraPose\cite{su2022zebrapose}或GDR-net\cite{wang2021gdr},仅需少量修改即可输出可见掩码和完整掩码。
