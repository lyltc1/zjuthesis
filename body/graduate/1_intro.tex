\chapter{引言}
\par 目标物体的位姿估计问题是一个视觉感知问题,是机器人应用中重要的基础问题。只有机器人系统实现目标物体的定位,才能够实现环境交互,例如机械臂实现物体抓取、物体操作以及避障和跟踪等下游任务。目前,操作机器人还无法广泛应用于非工业场景,主要原因便是对于环境理解的能力不足,无法快速适应非结构化甚至未知环境。因此,解决位姿估计问题能够提高操作机器人的环境感知能力,提高操作机器人在非结构化环境中的快速部署能力。
\section{研究背景与意义}
\par 智能机器人是国家的重要战略产业,是未来人类社会发展的重要方向,具有重要的社会意义和经济价值。智能机器人的运行过程可以分为三个阶段:感知环境,规划行为以及执行操作。对环境的感知能力是限制智能机器人发展的一个主要原因。
\par 智能机器人、智能制造、智能物流等是国家重大需求,已经列入国家多个科技发展计划,具有重要社会意义和经济价值。机器人技术的发展推动社会的变革并带来行业的竞争,在我国《中国制造 2025》计划~\cite{中国制造2025} 和德国提出的工业 4.0~\cite{德国工业4.0}中,机器人科技作为“十大技术领域”被列出。
\par 继自动驾驶领域的快速发展,近两年来国内外具身智能公司发展迅猛,众多企业纷纷布局。从美国的波士顿动力(Boston Dynamics)公司的电驱Atlas机器人到特斯拉公司(Telsa)开发的擎天柱(Optimus)人型机器人再到李飞飞创办的空间智能公司World Labs,美国正走在具身智能的行业前沿。放眼国内,优比选、宇树、智元以及多个国家人形机器人创新中心俨然成为行业标杆,更有傅里叶智能、星海图、星尘智能等等数十家初创公司崭露头角。
\par 近两年来,大语言模型(Large Language Model)和基于图片-文本对训练的多模态模型在自然语言处理和计算机视觉领域取得了显著进展。例如OpenAI公司开发的GPT-4o模型,在语言理解、生成等多个领域表现出色,能够在多轮对话中完成上下文理解并输出高质量文本。
\par 本文以机器人对目标物体进行抓取作为研究背景,研究基于深度学习的目标物体的六自由度位姿估计(6-DoF Object Pose Estimation)方法,旨在提升机器人在各种环境下对刚性物体的视觉感知能力。
\par 目前机器人操作领域的研究百花齐放。OpenVLA~\cite{openvla}和RDT~\cite{liu2024rdt}等工作基于视觉-文本-机器人行为的多模态数据进行训练,实现了通过人类指令输入,机器人通过观测图像能够直接输出对应的动作。这类技术有极大的发展前景,但目前仍然受限于数据规模和数据质量,成功率有待提升,并且无法完成精细化的抓取操作。Unidexgrasp++~\cite{wan2023unidexgrasp++}和Yuyang Li等~\cite{li2024grasp} 采用强化学习路线学习机器人抓取动作,目前这类方法以成功抓取物体为目标,能够在仿真环境训练并在真实环境中部署,但无法准确控制目标物体的位姿。基于深度学习的位姿估计方法~\cite{hodan2024bop}利用大量的数据进行训练,能够在真实环境中准确估计目标物体的位姿,并且搭配轨迹规划算法,实现精细化的抓取操作。
\subsection{机器人抓取领域的应用}
\subsection{其他领域的应用}
\subsubsection{一个 Subsubsection}

\section{}
