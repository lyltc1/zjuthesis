\section{引言}
\par 与结合深度信息\cite{9191119, 9429889, 9933183, 9206072, 2024hipose}的方法相比,仅使用RGB的方法\cite{8765585, billings2019silhonet, hybridpose}提供了更广泛的应用范围和更低的成本。

目前,高性能方法\cite{wang2021gdr, su2022zebrapose}依赖在2D图像像素和3D物体表面顶点之间建立对应关系。然后使用某种Perspective-n-Point (PnP)算法\cite{EPnP}变体来估计位置和姿态。对应关系通常以两种方式使用:(1) 作为网络输出的目标\cite{su2022zebrapose},随后应用PnP算法来估计位姿,或(2) 作为中间变量辅助网络学习,最终通过网络输出物体的位姿\cite{wang2021gdr}。然而,PnP算法需要2D和3D点之间的一对一映射,这对对称物体来说是一个挑战。例如,在一个无纹理的球体的情况下,图像中的任何像素都可以对应到物体表面上的任何点。当目标对应关系由单一的真值位姿指定时,问题变得更加突出。

先前的工作\cite{pvnet, park2019pix2pose, su2022zebrapose}通过使用RANSAC-PnP算法剔除稠密对应关系中的外点来缓解对称性歧义的问题。但这种缓解对称性歧义的方式将会浪费大量正确但相互间不一致的对应关系。其他工作\cite{wang2021gdr, di2021so}训练端到端网络直接回归位姿参数,绕过PnP的求解问题,但训练一对一对应关系的步骤仍然没有解决对称性歧义的问题。在实验中发现,当遮挡情况不严重时,网络强大的拟合能力能够选择一种一致的对应关系进行输出。但在严重遮挡的情况下,会出现网络输出混乱的问题,对应关系预测的准确性显著下降,进而使得姿态估计精度显著下降。

\par 为解决这一问题,提出了SymCode——一种密集对应编码方法。该方法通过建立一对多的2D-3D对应关系进行编码,能够适用于各种对称形式的物体,也能够适用于多种对称形式混合的物体。对于非对称物体,这种一对多的对应关系将退化为传统的一对一对应模式。

\par 受ZebraPose\cite{su2022zebrapose}表面编码方法的启发,通过为每个对应关系分配离散编码来优化网络训练。SymCode在保留对称性信息表征能力的同时,突破了PnP算法对一对一对应关系的固有假设限制。SurfEmb\cite{haugaard2022surfemb}同样面临类似挑战,其通过P3P算法生成大量候选位姿并经过精细优化获得最终结果,但该过程耗时严重。相比之下,提出的SymNet端到端网络可直接从SymCode编码中输出目标位姿及分割掩码,显著提升计算效率。

\par \autoref{compare_encoder}展示了多种对应关系的表示方法,其中,a图为物体图像,b图为无纹理模型,c、d、e分别为3D坐标编码、ZebraPose编码\cite{su2022zebrapose}以及提出的SymCode。基于一对一对应关系的3D坐标编码和ZebraPose编码没有考虑对称性。相比之下,基于一对多对应关系的SymCode明确保留了对称性信息。

\begin{figure}[htbp]
    \centering
    \begin{overpic}[width=0.9\textwidth]{figure/symnet/compare_encoder.jpg}
        \put(9,40){\small (a)}
        \put(28,40){\small (b)}
        \put(47,40){\small (c)}
        \put(67,40){\small (d)}
        \put(88,40){\small (e)}
        \put(5,2){输入图像}
        \put(22,2){无纹理模型}
        \put(44,2){3D坐标}
        \put(60,2){ZebraPose编码}
        \put(82,2){SymCode编码}
    \end{overpic}
    \caption{不同表面编码对比图}
    \label{compare_encoder}
\end{figure}

本文的核心贡献可归纳为以下三个方面:
\begin{enumerate}
\item 提出SymCode对称感知二进制表面编码方法,通过一对多2D-3D对应关系实现位姿估计;
\item 设计SymNet网络架构,无需依赖PnP-RANSAC方法即可从一对多对应关系中估计物体位姿,在推理时间上实现显著优化;
\item 基于T-LESS和IC-BIN等包含对称性物体的数据集,系统研究物体对称性对位姿估计的影响,并验证不同对称感知方法的性能表现。
\end{enumerate}

\section{对称性歧义}

\par 对称性歧义是指物体在不同位姿下的成像完全相同的现象,导致无法唯一确定其真实位姿,也称为位姿歧义。通常在物体位姿估计领域,具有对称性歧义的物体也简化称为对称物体。注意,对称性歧义与物体具有对称性不是等价的概念,需要物体能够在不同位姿下产生相同的成像。

\par \autoref{fig:对称性物体}展示了几类常见的具有对称性歧义的物体,包括工业零件,球类,餐具等。对称性歧义也可能是由于遮挡导致关键信息缺失而无法唯一确定物体的位姿。

\begin{figure}[htbp]
    \centering
    \begin{overpic}[width=1.0\textwidth]{figure/symnet/对称性物体.jpg}
    \end{overpic}
    \caption{具有对称性歧义的物体}
    \label{fig:对称性物体}
\end{figure}

\par 直接法需要折中所有可能的位姿,有时甚至输出平均位姿作为解决方案,但这种位姿可能并非有效解。早期研究如SSD6D\cite{ssd6d}和BB8\cite{rad2017bb8}通过约束旋转角度范围来消除训练集中的位姿歧义。Pitteri等人\cite{pitteri2019object}提出将对称位姿映射为同一表示的通用解析方法,但会导致映射后位姿的不连续性,可能引发网络输出的突变现象。CosyPose~\cite{labbe2020cosypose}采用对称感知的ADD-S损失函数,通过将物体最接近的对称位姿作为真值来评估位姿精度。ES6D~\cite{mo2022es6d}将对称性划分为五种类别,并据此提出对称不变的A(M)GPD损失函数,其性能优于传统ADD(S)损失。然而,该方法难以直接应用于基于对应关系的方法。

\par 近期基于对应关系的方法通常以一对一对应关系作为辅助学习目标。但对称性会引发对应关系的歧义性,导致图像像素在模型表面存在多个潜在对应顶点,从而产生位姿歧义。Jesse Richter-Klug等人\cite{richter2021handling}提出"闭合对称环"表示方法,通过编码不同点与物体原点间的角度信息实现一对多对应关系,经信息融合后最终得到一对一对应。EPOS~\cite{hodan2020epos}利用表面片段处理对称性,建立一对多2D-3D对应关系,但最终仍需通过P3P求解器从采样的三点对应中估计位姿,在连续对称场景下对片段坐标系的三维坐标回归仍存在局限。GDR-Net\cite{wang2021gdr}通过计算最接近对称位姿的损失指导位姿回归,同时继承了EPOS\cite{hodan2020epos}的表面片段方法。SurfEmb\cite{haugaard2022surfemb}以自监督方式学习一对多对应关系,但需从中采样一对一对应关系以适应RANSAC-P3P算法的输入要求。据我们所知,本研究是首个直接从一对多对应关系中估计位姿而绕过PnP的物体位姿估计工作。