\chapter{总结与展望}

\section{本文总结}

\par 本文针对复杂场景下6D物体位姿估计的鲁棒性与实时性挑战,从RGB-D与RGB两种模态切入,提出了一系列创新性方法,改进了面向密集对应关系的位姿估计框架。研究内容与创新点如下:
\par (1)基于RGB-D模态,提出层次化表面编码与对应关系剪枝机制。通过层次化二进制编码策略,从粗到精逐步确立点云与物体表面对应关系,结合离群点动态剔除技术,在保证精度的同时提升计算效率,解决了传统方法在复杂场景下的鲁棒性瓶颈。
\par (2)针对RGB与深度数据互补性利用不足的问题,设计双向一致性融合网络,通过双分支协同学习机制同步提取表观特征与几何特征。引入像素-观测点联合损失函数与跨模态一致性约束,实现多模态信息的深度融合,显著提升了对称物体及弱纹理目标的位姿估计精度。
\par (3)针对RGB模态下对称物体位姿歧义问题,提出基于一对多对应关系的对称感知表面编码方法,结合直接回归6D参数的端到端网络,突破传统PnP求解的局限性,在保持实时性的同时将对称物体位姿误差降低,同时不影响对非对称物体位姿估计的性能。
\par (4)提出基于可见/完整掩码的轮廓对齐优化方法,通过几何一致性误差函数迭代优化位姿参数。与传统像素级优化方法相比,仅需渲染物体轮廓即可实现精度提升,计算效率提高,为嵌入式设备部署提供了可行性。
\par 本文系列成果为机器人视觉伺服、工业抓取等实际应用提供了高精度、低延迟的位姿估计解决方案,同时建立了多模态融合与对称处理的理论范式。未来研究将进一步探索无模型估计与动态场景下的连续位姿跟踪问题。

\section{未来展望}
\par 实例级物体位姿估计作为当前工业落地的核心技术之一,其研究突破仍需聚焦实际场景需求,而泛化物体位姿估计(开放世界场景的位姿估计)作为领域长期发展目标,两者在技术路线上呈现渐进式关联。未来研究可从实例级位姿估计的深化方向与向泛化位姿估计的拓展路径两方面展开。
\par 针对实例级位姿估计的深化方向,当前高精度方法需针对单个物体进行小时级训练,难以满足工业快速部署需求。未来可通过元学习框架或预训练-微调范式,利用跨物体几何先验知识实现参数快速迁移,结合轻量化动态网络架构设计,在分钟级训练时间内保持模型精度。此外,探索自监督几何一致性约束替代全监督训练,有望进一步缩短训练周期。现有“一物体一模型”范式在同时训练多个物体时性能显著下降,需研究多物体共享表征学习,通过可扩展的动态网络结构实现模型容量弹性扩展,并结合物体间几何关系解耦模块,利用图神经网络建模类别拓扑关联,缓解特征干扰问题。
\par 针对泛化位姿估计的拓展路径,作为实例级技术的自然延伸,泛化位姿估计需突破以下瓶颈以释放更大应用潜力:针对当前方法推理耗时较长的问题,需优化网络结构与推理流程,设计高效的轻量化模型,同时结合模型剪枝与量化技术,提升推理速度以满足实时性需求。此外,需研究动态场景下的连续位姿跟踪问题,结合时序信息与运动先验,设计鲁棒的时空一致性约束优化方法,解决遮挡与目标丢失问题。泛化位姿估计的研究将为机器人在开放世界中的自主操作提供技术支撑,推动智能系统向更高层次的感知与交互能力发展。
\par 实例级研究的核心成果将为泛化方向提供关键技术组件,而泛化位姿估计的突破又将反哺实例级方法在数据生成、模型预训练等环节的效能提升。二者协同发展将推动位姿估计技术在智能制造(柔性产线多物体抓取)、服务机器人(开放场景物体操作)等领域的深度融合应用,最终构建通用化的物体级环境感知系统。

