\chapter{基于表面编码和对应关系剪枝的RGB-D物体位姿估计方法}
\inputbody{2_1_hipose_introduction.tex}
\inputbody{2_2_hipose_encoding.tex}
\inputbody{2_3_hipose_method.tex}
\inputbody{2_4_hipose_network.tex}
\inputbody{2_5_hipose_experiment.tex}

\section{本章小结}
\par 本章提出了一种基于单张 RGB-D 图像的实例级物体 6 自由度位姿估计方法 HiPose。该方法通过层次化表面编码与对应关系剪枝机制,从粗到精逐步建立观测点云与物体表面区域的对应关系。具体而言,HiPose 首先利用大尺度表面区域进行初始匹配以降低计算复杂度,随后通过迭代优化逐步收缩匹配区域,实现精确点对点匹配。同时,HiPose 在迭代过程中动态剔除离群点,从而在保证位姿精度的同时显著提升了算法效率。实验结果表明,该方法在多个基准数据集上均表现出优异的性能。